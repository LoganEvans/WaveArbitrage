\documentclass{article}



\usepackage{arxiv}

\usepackage[utf8]{inputenc} % allow utf-8 input
\usepackage[T1]{fontenc}    % use 8-bit T1 fonts
\usepackage{hyperref}       % hyperlinks
\usepackage{url}            % simple URL typesetting
\usepackage{booktabs}       % professional-quality tables
\usepackage{amsfonts}       % blackboard math symbols
\usepackage{amsmath}
\usepackage{nicefrac}       % compact symbols for 1/2, etc.
\usepackage{microtype}      % microtypography
\usepackage{graphicx}
\usepackage{natbib}
\usepackage{doi}
\usepackage{csquotes}


\def \thetitle {Wave Arbitrage: A Disproof of the Efficient Market Hypothesis}
\title{\thetitle}

\date{February 12, 2021}

\author{
  \href{https://orcid.org/0000-0001-6450-3262}{\includegraphics[scale=0.06]{orcid.pdf}\hspace{1mm}Logan P.~Evans}
  \\ \texttt{loganpevans@gmail.com}
}

% Uncomment to override  the `A preprint' in the header
%\renewcommand{\headeright}{Technical Report}
%\renewcommand{\undertitle}{Technical Report}
\renewcommand{\shorttitle}{\textit{arXiv} Template}

%%% Add PDF metadata to help others organize their library
%%% Once the PDF is generated, you can check the metadata with
%%% $ pdfinfo template.pdf
\hypersetup{
pdftitle={\thetitle},
pdfsubject={econ.EM, econ.TH},
pdfauthor={Logan P.~Evans},
pdfkeywords={Efficient Market Hypothesis, Wave Arbitrage},
}

\newtheorem{theorem}{Theorem}
\renewcommand\thetheorem{\unskip}

\newtheorem{corollary}{Corollary}
\renewcommand\thecorollary{\unskip}

\newtheorem{lemma}{Lemma}
\renewcommand\thelemma{\unskip}

\begin{document}
\maketitle

\begin{abstract}
  Wave arbitrage is a trading strategy that has a higher expected return rate
  than the buy-and-hold strategy when the market model is a martingale. This
  is a contradiction to the efficient market hypothesis. We compare the wave
  arbitrage and buy-and-hold strategies for SP500 stocks from late 2016 to
  early 2021 and demonstrate that wave arbitrage produces excess returns, even
  accounting for trading fees.
\end{abstract}

% keywords can be removed
\keywords{Efficient Market Hypothesis \and Wave Arbitrage}

\section{Introduction}
The efficient market hypothesis has been a useful way to think about markets for
over fifty years. According to \citet{fama1970},

\begin{displayquote}
A market in which prices always “fully reflect” all available information is called “efficient.”
\end{displayquote}

There are several tests that can show that a market is inefficient.
\citet{fama1976} summarized one of those tests with the observation, "... in an
efficient market, trading rules with abnormal returns do not exist.” In other
words, if the efficient market hypothesis is true, it is impossible to beat the
buy-and-hold strategy except through luck.

Wave arbitrage is similar to tidal power generators. As ocean waves come in
or go out, the water turns a water wheel and generates electricity. With wave
arbitrage, when the stock price moves up or down, the algorithm harvests some
of the energy of that wave by selling when the price rises and buying when the
price falls.

The wave arbitrage strategy will only work if the market fluctuates. However,
fluctuation is a defining attribute of the stock market. New information causes
a market to wobble as traders react to that news, and there will always be more
news. According to \citet{brooks2014}, an acquaintance of J. P. Morgan once
asked him what the market would do, to which Morgan replied, “It will
fluctuate.”

\section{Wave Arbitrage}
Consider a scenario where a trader is fully invested in two companies and at
time $t$ must decide the proportion of their assets invested in each share.

In some situations it’s easy to determine whether a trader’s portfolio is on
the correct side of a trade. If the trader is holding only shares of stock $i$
and the price of that stock $p_i$ increases while $p_j$ does not increase,
then the trader is on the correct side of the trade. However, what about
situations where the trader is not fully invested in one stock? We use a
geometric mean to decide whether a portfolio comes out ahead or behind on any
individual event.

Let $s_i$ represent the number of shares in security $1$. The value of the
portfolio, using $s_i$ as the basis of measurement, is
$s_i + \frac{s_j p_j}{p_i}$. However, since both $s_i$ and $s_j$ are both
reasonable bases for measurement, we take the geometric mean of the two
representations and obtain

\begin{equation}
\label{G_definition}
  G = \sqrt{ \bigg( s_i + \frac{s_j p_j}{p_i} \bigg)
             \bigg( s_j + \frac{s_i p_i}{p_j} \bigg)
           }
    = \frac{s_i p_i + s_j p_j}{\sqrt{p_i p_j}}
\end{equation}

Let $\alpha_i$ represent the proportion of the portfolio invested in security
$i$. At time $t$, the prices $p_i$ are constant, so $G$ has the property that
for any $\alpha_i$, $G$ is a constant value. More plainly, if a trader can
ignore exchange fees, then the value of $G$ is the same both before and after
each transaction.

Outside of rebalance events, $G$ is a function of the prices $p_i$. To find the minimum, we take the derivative with respect to $p_i$:

\begin{equation}
\begin{aligned}
  G &= \frac{s_i p_i + s_j p_j}{\sqrt{p_i p_j}} \\
  \frac{dG}{dp_i} &= \frac{p_j (s_i p_i - s_j p_j)}{2 (p_i p_j)^{\frac{3}{2}}} \\
\end{aligned}
\end{equation}

Assuming prices are always greater than zero, the only root is at
\begin{equation}
\label{eq:balance}
  s_i p_i = s_j p_j .
\end{equation}

This represents the global minimum. This gives us the
following:

\begin{lemma}
\label{lemma}
  The value $G$ will monotonically increase if the only actions performed is to
  rebalance the portfolio it so that $\alpha_i = \frac{1}{2}$.
\end{lemma}

This observation leads to the wave arbitrage algorithm. Fluctuations in the
market will cause the value of $G$ to change, but the value will never be less
than $G_t$, the value of $G$ after rebalancing $\alpha_i = \frac{1}{2}$ at
time $t$. The trigger for rebalance events can be selected so that
$G_{t+1} > G_t$ accounting for any transaction fees.

If the portfolio is holding cash as one of its two assets, then
\ref{eq:balance} simplifies to $d = s p$. This leads to

\begin{corollary}
  If price $p$ is observed at times $t$ and $t+1$, then if the only
  transactions made to the portfolio are to rebalance cash and shares to equal
  ratios, then the value of the portfolio will be strictly greater at time
  $t + 1$ if at least one rebalance event has occurred.
\end{corollary}

\section{Disproof of the Efficient Market Hypothesis}

\bibliographystyle{unsrtnat}
\bibliography{references}

\end{document}
