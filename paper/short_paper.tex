\documentclass[10pt]{article}

\usepackage{amsfonts}       % blackboard math symbols
\usepackage{amsmath}
\usepackage[margin=0.25in]{geometry}
\usepackage{multicol}
\usepackage{xfrac}          % compact symbols for 1/2, etc.
\usepackage{natbib}

\def \thetitle {Martingale Markets are Inefficient}
\title{\thetitle}

\author{Logan P.~Evans \texttt{(loganpevans@gmail.com)}}

\date{}

\newtheorem{theorem}{Theorem}
\newtheorem{corollary}{Corollary}
\newtheorem{lemma}{Lemma}

\begin{document}
\begin{multicols}{2}
\maketitle

\begin{abstract}
  \noindent
  We present the $G$-score for a portfolio and develop a trading strategy for
  which this value monotonically increases. We show that the expected value of
  this algorithm has a higher expected value than a buy-and-hold strategy,
  which is a contradiction to the Efficient Market Hypothesis.
\end{abstract}

\section{Introduction}

The Efficient Market Hypothesis (EMH) has been a useful way to think about
markets for over half a century. According to \citet{fama1970}, ``A market in
which prices always “fully reflect” all available information is called
`efficient'.''

\section{Wave Arbitrage}

Consider a scenario where a trader is fully invested in two equities and at
time $t$ must decide the proportion of their assets invested in each company.
Let $s_i$ represent the number of shares a portfolio holds of security $i$. The
value of the portfolio, using $s_i$ as the basis of measurement, is $S_i = s_i
+ \frac{s_j p_j}{p_i}$. However, since both $S_i$ and $S_j$ are both
representations for the value of a portfolio, we take the geometric mean of the
two and obtain the $G$-score:

\begin{equation}
\label{eq:g_def}
  G = \sqrt{ \bigg( s_i + \frac{s_j p_j}{p_i} \bigg)
             \bigg( s_j + \frac{s_i p_i}{p_j} \bigg)
           }
    = \frac{s_i p_i + s_j p_j}{\sqrt{p_i p_j}}.
\end{equation}

Let $\alpha_i$ represent the proportion of the portfolio invested in security
$i$. At time $t$, the prices $p_i$ are constant, so $G$ has the property that
for any $\alpha_i$, $G$ is a constant value. More plainly, if a trader can
ignore exchange fees, then the value of $G$ is the same both before and after
any trade.

Outside of rebalance events, $G$ is a function of the prices $p_i$ and $p_j$.
To find the minimum, we take the derivative with respect to $p_i$:

\begin{equation}
\begin{aligned}
\label{def:G}
  G &= \frac{s_i p_i + s_j p_j}{\sqrt{p_i p_j}} \\
  \frac{dG}{dp_i} &= \frac{p_j (s_i p_i - s_j p_j)}{2 (p_i p_j)^{\frac{3}{2}}} \\
\end{aligned}
\end{equation}

Assuming prices are always greater than zero, the only root is at
\begin{equation}
\label{eq:balance}
  s_i p_i = s_j p_j .
\end{equation}

This represents the global minimum. This leads to

\begin{lemma}
\label{lemma}
  The sequence $\{G\}$ will monotonically increase if the only actions performed
  is to rebalance the portfolio so that $\alpha_i = \frac{1}{2}$.
\end{lemma}

We refer to this algorithm as ``wave arbitrage'' because it works by harvesting
energy from waves in stock prices.

\section{Disproof of the EMH}
\label{sec:disproof}

\citet{fama1970} describes a test that can show that a market is inefficient.
The total excess market value at $t+1$ is

\begin{equation}
  V_{t+1}
    = \sum_{j=1}^n \alpha_j (\Phi_t) [ r_{j,t+1} - E(\tilde{r}_{j,t+1} | \Phi_t)].
\end{equation}

The value $\alpha_j (\Phi_t)$ represents the proportion of assets invested in
security $j$, $r_{j,t+1}$ is the actual return at $t+1$, and
$E(\tilde{r}_{j,t+1} | \phi_t)$ is the expected equilibrium returns given the
available information $\Phi_t$. Using the ``fair game'' property, meaning that
price movements follow a martingale, he concludes

\begin{equation}
\label{eq:excess_returns}
  E(\tilde{V}_{t+1} | \Phi_t)
    = \sum_{j=1}^n \alpha_j (\Phi_t) E(\tilde{z}_{j,t+1} | \Phi_t)
    = 0.
\end{equation}

In other words, no strategy will outperform any other strategy. Since the
buy-and-hold strategy avoids all trading costs, it can be used as a benchmark.

Since Lemma \ref{lemma} implies that $\{G\}$ monotonically increases under wave
arbitrage, the goal is to determine whether the expected value of the portfolio
also increases.

Inspecting the expected value of wave arbitrage, we let
$S_i = s_i + \frac{s_j p_j}{p_i}$ represent the potential number of shares of
security $i$ that a portfolio can produce if all shares of security $j$ are sold
and the proceeds are used to buy shares of security $i$. Since $G$ is the
geometric mean of two quantities, we can use the inequality of arithmetic and
geometric means to write

\begin{equation}
  G = \sqrt{S_i S_j} \leq \frac{S_i + S_j}{2}.
\end{equation}

When using wave arbitrage, since the sequence $\{G\}$ grows monotonically and
diverges due to Lemma \ref{lemma}, the arithmetic mean of $S_i$ and $S_j$ will
also diverge. From this we conclude that the expected number of potential
shares increases:

\begin{equation}
  E(S_{i,t+1}) > E(S_{i,t}).
\end{equation}

Under the buy-and-hold strategy, the number of shares held does not change, but
$S_i$ is a function of the prices and can change. However, the expected returns
from security $i$ is equal to the expected returns from security $j$; if that
were not the case, an investor could allocate all of their assets to one or the
other of the securities and have an abnormally high expected return. This
implies that the expected price change $\delta$ is equal for the two assets.
However, if $E (p_{i,t+1}) = \delta p_{i,t}$ and $E (p_{j,t+1}) = \delta
p_{j,t}$, then

\begin{equation}
  E(S_{i,t+1}) = s_i + \frac{s_j (\delta p_{j,t})}{\delta p_{i,t}} = E (S_{i}).
\end{equation}

Therefore, the expected return rate for wave arbitrage is higher than the
expected return rate for buy-and-hold and Equation \ref{eq:excess_returns} is
false.

\section{Conclusion}
Martingale markets are inefficient.

\bibliographystyle{unsrtnat}
\bibliography{references}

\end{multicols}
\end{document}
